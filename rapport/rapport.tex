\documentclass[11pt]{report}

\usepackage[utf8]{inputenc}
\usepackage[francais]{babel}
\usepackage[T1]{fontenc}
\usepackage{graphicx}

\title{ Projet LO41 : Les boucles de recompletement}
\author{Chiara \bsc{Salvoni} \& Romain \bsc{Thibaud}}
\date{Automne 2014}

\renewcommand{\contentsname}{Sommaire}

\begin{document}

\maketitle

\tableofcontents

\chapter{Introduction}

Ce projet a été réalisé dans le cadre d'un enseignement du département informatique de l'UTBM. Il a pour objectif de nous faire accuérir des connaissances en programmation orienté système. Le choix du langage C a été imposé afin qu'il ne soit question que d'une application à un nouveau problème des différentes notions abordées dans ce même langage dans le cadre des travaux pratiques. Ce projet constitue une traduction en langage informatique d'une méthode de production pratiquée dans le monde industriel : la méthode Kanban. 

Il est d'abord nécessaire de décrire concrètement en quoi consiste cette méthode et dans quelle mesure nous devons l'implémenter. 

S'en suit alors une analyse du problème posé afin d'y répondre aux mieux. Nous utlisons par exemple un réseau de pétri afin d'imager la problématique. 

Enfin il ne reste plus qu'à implémenter en C une solution en choisissant parmi les outils que nous avons vu en cours et s'organiser pour effectuer un travail à plusieurs.

\chapter{La méthode Kanban : description}

	\section{Description}
	
	La méthode Kanban est une méthode de production inventé par les ingénieurs de chez Toyota dans les années 50. Elle tranche avec les méthodes utilisées précédemment par sa volonté de réduire la quantité des stocks et donc de ne produire que le strict nécessaire. La production s'adapte donc à la consommation en temps réels. Cette vision de la production facilite un diversification des productions et rompt donc avec la standardisation engendré par des méthodes tel le Fordisme. Elle garde néammoins un aspect Tayloriste puisqu'elle continue de disséquer la fabrication du produit pour en optimiser le temps nécessaire à sa création.
	
	Cet outils de productions repose sur un système d'étiquettes (Kanban en japonnais) qui permet la synchronisation des différents postes entre eux. 
	\section{Donnée du problème}
	
\chapter{Analyse du problème}

	\section{Observations globales}
	
	\section{Réseau de pétri}
		
\chapter{Mis en \oe{}uvre}

	\section{Organisation}
	
	\section{Choix techniques}
	
\chapter{Conclusion}
\end{document}